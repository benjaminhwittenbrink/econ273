\documentclass{article}
\usepackage{graphicx}
\usepackage{amsmath, amssymb, amsthm}
\usepackage{epsf}
\usepackage{bm}
\usepackage[parfill]{parskip}
\usepackage{mathptmx}
\usepackage{geometry}
\usepackage{fancyhdr}

\newcommand{\mc}[1]{\mathcal{#1}}
\newcommand{\mbf}[1]{\mathbf{#1}}
\def\E{\mathbb{E}}


\renewcommand{\thesection}{\Roman{section}}
\renewcommand{\thesubsection}{\Alph{subsection}}

\newcommand{\params_J}{100}
\newcommand{\params_J}{100}
\newcommand{\params_edu_types}{['H', 'L']}
\newcommand{\params_race_types}{['White', 'Black']}
\newcommand{\params_gamma_HH}{-0.2}
\newcommand{\params_gamma_HL}{0.5}
\newcommand{\params_gamma_LL}{-0.5}
\newcommand{\params_gamma_LH}{0.7}
\newcommand{\params_alpha_HH}{0.8}
\newcommand{\params_alpha_HL}{0.1}
\newcommand{\params_alpha_LH}{0.2}
\newcommand{\params_alpha_LL}{0.7}
\newcommand{\params_iota}{0.05}
\newcommand{\params_phi}{1}
\newcommand{\params_phi_geo}{0.3}
\newcommand{\params_phi_reg}{0.4}
\newcommand{\params_phi_a}{0.5}
\newcommand{\params_zeta}{0.8}
\newcommand{\params_beta_st}{3}
\newcommand{\params_beta_st}{3}
\newcommand{\params_beta_x}{{'White': 0.3, 'Black': 0.7}}
\newcommand{\params_beta_a}{{'White': 0.3, 'Black': 0.9}}
\newcommand{\params_beta_w}{{'White': 0.9, 'Black': 0.2}}
\newcommand{\params_H}{{'Black': {'N': 20}, 'White': {'N': 80}}}
\newcommand{\params_L}{{'Black': {'N': 180}, 'White': {'N': 20}}}
\newcommand{\params_Z_H}{{'mu': 0, 'sigma': 0.5}}
\newcommand{\params_Z_L}{{'mu': 0, 'sigma': 0.5}}
\newcommand{\params_x_geo}{{'mu': 0, 'sigma': 0.5}}
\newcommand{\params_x_reg}{{'mu': 0, 'sigma': 0.5}}
\newcommand{\params_CC}{{'mu': 0, 'sigma': 0.1}}
\newcommand{\params_x}{{'mu': 0, 'sigma': 0.1}}
\newcommand{\params_epsilon_H}{{'mu': 2, 'sigma': 0.1}}
\newcommand{\params_epsilon_L}{{'mu': 1, 'sigma': 0.1}}
\newcommand{\params_epsilon_a}{{'mu': 0, 'sigma': 0.1}}

\newcommand{\paramsestbetawWhite}{0.8995}
\newcommand{\paramsestbetaaWhite}{0.3019}
\newcommand{\paramsestbetawBlack}{0.1979}
\newcommand{\paramsestbetaaBlack}{0.9022}
\newcommand{\paramsestbetast}{3.0096}
\newcommand{\paramsestalphaHH}{0.8046}
\newcommand{\paramsestalphaHL}{0.0992}
\newcommand{\paramsestgammaHH}{-0.2117}
\newcommand{\paramsestgammaHL}{0.5241}
\newcommand{\paramsestalphaLH}{0.1916}
\newcommand{\paramsestalphaLL}{0.7019}
\newcommand{\paramsestgammaLH}{0.7264}
\newcommand{\paramsestgammaLL}{-0.5337}
\newcommand{\paramsestiota}{0.0501}
\newcommand{\paramsestphi}{0.9971}
\newcommand{\paramsestphigeo}{0.3007}
\newcommand{\paramsestphireg}{0.4012}
\newcommand{\paramsestphia}{0.4983}



\pagestyle{fancy}
\fancyhead[L]{\textsc{Bäcker-Peral, Kelly, and Wittenbrink}}
\fancyfoot[C]{\thepage} % page number in the center of the footer
\renewcommand{\headrulewidth}{0pt} % remove header rule if desired
\renewcommand{\footrulewidth}{0pt} % remove footer rule if desired
% read in parameters


\title{14.273 IO Mini-Project}
\author{Benjamin Wittenbrink, Jack Kelly, and Verónica Bäcker-Peral}

\begin{document}

\maketitle

\section{Motivation}

Over the past decades, the wage gap between high school and college graduates has significantly widened, accompanied by a pronounced geographic sorting of workers by skill. Metropolitan areas with a high concentration of college-educated residents have further increased their share of such workers between 1980 and 2000. This phenomenon, often referred to as ``the Great Divergence," has seen high-skill cities experience both robust wage growth and soaring housing costs. Such trends raise important questions about whether higher wages for college graduates have translated into improved economic well-being, given the offsetting effects of elevated local prices. While higher housing (and lifestyle) costs may diminish consumption benefits, the attractive amenities in these cities could potentially enhance overall worker welfare. In this project, we exposit, simulate, and estimate the empirical model in Diamond (2016): ``The Determinants and Welfare Implications of US Workers' Diverging Location Choices by Skill: 1980-2000." Here, we present the model, with slight simplifications, that we will use for simulation and estimation. 
In particular, the paper uses a structural spatial equilibrium model to examine the determinants of this geographic sorting of workers by skill and the corresponding welfare implications. The key feature is that wages, prices, amenities are all codetermined by the balance of high and low skilled workers who endogenously move to a given area. 

\section{Model}

Our model closely follows Diamond, with the major changes being that we collapse the model down to one time period, and simplify it to have one endogenous and one exogenous amenity. 

We model three markets that clear simultaneously: labor, housing, and an endogenous amenity. There are a set of locations $J$, and a set of workers $i$ who vary on three dimensions: their education level ($\in \{W,H\}$), their race ($\in \{White,Black\}$), and their  location $j$ of origin. 


\subsection{Labor Demand}

We assume, following Diamond, the following log-linear labor demand curves for high and low education types: 
\begin{align}
    w_{j}^H &= \gamma_{HH}\log H_{j} + \gamma_{HL} \log L_{j} + \varepsilon_{j}^H \label{eq_wage_h_aprx} 
    \\ 
    w_{j}^L &= \gamma_{LL}\log L_{j} + \gamma_{LH} \log H_{j} + \varepsilon_{j}^L \label{eq_wage_l_aprx} 
\end{align}
 where $w_j^e$ represents log wages for each education level. These equations can be microfounded as approximating a model with homogeneous firms with a  production technology that is Cobb-Douglas in capital and a labor aggregator, which is CES between high and low skill types; details of the microfoundation are in Appendix A. Importantly, in this model the elasticities $\gamma$ may capture both classical labor demand forces  -- in which higher wages lead to lower quantities demanded of labor -- and ``spillover" effects whereby the mix of high and low skill types affects endogenous productivity. 

\subsection{Labor Supply and Amenity Demand }


To focus on the geographic sorting of workers, the model abstracts from endogenous labor supply in the sense of choosing hours worked or occupation type. Rather, the sole dimension of labor supply choice is where to live and work. Since this also depends on the level of amenities in a given location, it further serves as the demand side of our amenity market.

In particular, we posit population shares of the form 
\begin{align}\label{eq: population shares} 
H_{j} &= \sum_{i\in\mc{H}} \frac{\exp(\delta_{jd}^H + \beta_d^{st}st_{ij})}{\sum_{k} \exp(\delta_{kd}^H + \beta_d^{st}st_{ik})} \\
L_{j} &= \sum_{i\in\mc{L}} \frac{\exp(\delta_{jd}^L + \beta_d^{st}st_{ij})}{\sum_{k} \exp(\delta_{kd}^L + \beta_d^{st}st_{ik})}
\end{align}

where  $\delta_{jd}^X$ as the utility value of the components of the city $j$ which all workers of race $d$ and education level $e$ value identically and $\st_{ij}$ corresponds to an indicator for whether a location is an individual's origin. The mean utilities take the form 
\begin{equation} \delta_{jd}^X = (w_{j}^e - \zeta r_{j})\beta_d^w + \beta^a_d a_{j} + \beta^x_d x_{j}^A \end{equation}

where $r_j, w_j^e$ are log rents and (education-specific) wages, $a_j$ is an endogenous amenity, and $x_j^{A}$ is an (unobserved) exogenous amenity. In Appendix A, we show how  these equations emerges from individuals choosing to spend their earnings on a national vs a local good (housing), and then making a discrete location choice based on the corresponding indirect utility and type one extreme value preference shocks. The parameters to be estimated are workers' preferences, some of which we allow to be education or race specific -- $\zeta, \beta_d^w, \beta_d^a$, and $\beta_d^{st}$ ($\beta_d^x$ is the coefficient on an unobservable, so without loss, it is 1).  

\subsection{Housing Demand}
Housing demand is given by 
\begin{equation} \label{eq: housing demand}  HD_{j}^* = L_{j}^* \frac{\zeta \exp(w_{j}^{L^*})}{\exp(r_{j}^*)} + H_{j}^* \frac{\zeta \exp(w_{j}^{H^*})}{\exp(r_{j}^*)} 
\end{equation}

Where $L_j^*, H_j^*$ come from the solutions of equations (3) and (4). The microfoundation is the same as the labor demand model, as detailed in Appendix A: individuals of education level $e$ choosing how to allocate their earnings housing versus the national good spend $\gamma \frac{\exp(w_j^e)}{\exp(r_j^e)}$ on housing, so total demand is given by the above. 



\subsection{Housing Supply}

Our housing supply equation is \begin{equation}\label{eq: housing supply}
    r_{j} = \log CC_{j} + \varphi_j \log HD_{j}.
\end{equation}

In Appendix A we show how this arises from a model of competitve housing supply with marginal costs that depend on contruction and land costs. The elasticity term $\varphi_j$ varies by geography and regulation
$$\varphi_j = \varphi + \varphi^{geo}x_j^{geo} + \varphi^{reg} x_j^{reg}$$
This captures the fact that responses to changes in housing demand will be heterogeneous by city and depend on geographic constraints $x_j^{geo}$ (e.g., mountains, lakes, etc.) and regulatory constraints $x_j^{reg}$ (e.g., zoning laws, etc.). 

\subsection{Amenity Supply}
The endogenous amenity is chosen by cities and is a function of the employment ratio,
\begin{equation}\label{eq: amenity supply} 
a_{j} = \varphi^a \log(H_{j}/L_{j}) + \varepsilon^a_{j} \end{equation}
where $\varphi^a$ is the elasticity of amenity supply. Here, $a_{j}$ is an index measuring the endogeneous amenities of a city. Thus, we allow this amenity index to respond to changes in the skill composition of a city. This setup is motivated by a rich literature showing that city amenities respond to the income levels of residents. 



\subsection{Equilibrium}
An equilibrium in this model is a collection of (skill-specific) wages, rents, and amenities, $(w^{L^*},w^{H^*},r^*,a^*)$ and skill-specific labor supply levels $(H_{j}^*,L_{j}^*)$ subject to the following market clearing conditions 
\begin{enumerate}
\item Labor markets clear for each skill level in each location: Equations (1) and (2) and Equations (3) and (4) hold. 
\item The housing market clears in each location: equations (6) and (7) hold.

\item The amenity market clears in each location: 
Equations (3), (4), and (8) hold. 
\end{enumerate}

\section{Simulation Procedure}

The model does not admit a closed form solution. Instead, we simulate the endogenous variables in our model through a simple fixed point procedure that uses our market clearing conditions. Table \ref{tab:parameter_values} lists our parameter values for our simulations. The broad strokes of our procedure are as follows: 
\begin{enumerate}
\item Draw exogenous variables given our parameters. These are housing supply, population demographics, amenities, and shocks to productivity and amenities. 
\item In the outer loop, we run a fixed point algorithm: for a given level of wages,  rents, and endogeneous amenities we can calculate the mean utility levels $\delta^z_{j}$ for each skill type $z$, which, together with the shocks, implies predicted population levels for each skill type according to equations (9) and (10). This in term implies new guesses of $w^H_{j}$ and $w^L_{j}$ using the labor supply equations (3) and (4), and implies a new level of amenities using the amenity supply equation (\ref{eq: amenity supply}).

\item In the inner loop, we find the conjectured rent level corresponding to the conjectured wage level by numerically solving equations \ref{eq: housing supply} and \ref{eq: housing demand} to yield an updated guess for rents. 
\end{enumerate}



\section{Estimation Procedure}
We assume that we observe instruments affecting skill-specific wages are independent of any remaining unobservable determinants. In particular, for each education level $e\in \{H,L\} $ we assume (and simulate)
 $$ w_{j}^e = \gamma_{eH}  \ln H_{j} + \gamma_{eL} \ln L_{j}+\alpha_{eH} Z_{j}^H  + \alpha_{eL} Z_{j}^L + \tilde{\varepsilon}_{jt}^e
 $$
where $Z_j^H,Z_j^L \perp \tilde{\epsilon}_j^e$. Our instrument consist of these $Z^e_j$ variables, interacted with the housing cost shifters $x_j^{geo}$ and $x_j^{reg}$. Our moment conditions are those implied by our structural equations. Formally, we have
$$\mathbb{E}[\tilde\varepsilon_j  Z_j]=0$$ 
where $Z_j \in \{ Z^X_{j},  Z_{j}^X x_j^{reg},  Z_{j}^X x_j^{geo}\}$  and $\varepsilon_j \in \{ \tilde\varepsilon_j^H, \tilde\varepsilon_j^L, \ln (CC_j), x_j^A, \varepsilon_j^a\}$. Appendix B derives each moment condition step-by-step. 

Our estimation procedure follows Diamond's. The procedure is roughly analogous to BLP, with the nonlinear parameter being the coefficient on staying in one's location of origin, $\beta_{st}$, since this is the only covariate that varies at the individual level conditional on education and race. The unobserved determinant of quality ($\xi$ in BLP-speak) is the unobserved amenity $x_j^A$. 

Our estimation procedure is as follows: 
\begin{enumerate}
\item In the outer loop, we conjecture values of the following parameters: $$\gamma_{HH}, \gamma_{HL}, \gamma_{LL}, \gamma_{LH}, \alpha_{HH}, \alpha_{HL}, \alpha_{LH}, \alpha_{LL},  \iota, \phi, \phi_{geo}, \phi_{reg}, \beta_{st}, \sigma_x.$$
\item In the inner loop, we solve for the unique education by race specific mean utilities $\delta_{jd}^e$ that rationalize equations (3) and (4) via a fixed point algorithm. In particular, starting from an initial guess of $\delta_{jd}^e(0) =0$, we update according to $\delta_{jd}^e(t+1)=\delta_{jd}^e(t) + \log(\hat{e}_j(\delta_{jd}^e(t), \beta_d^{st}) - \log(e_j^d)$  where $\hat{e}_j(\delta_{jd}^e(t)$ is the implied population share of education group $e$ and race $d$ given (3) and (4), and iterate until convergence. The Berry (1994) argument ensures that this procedure is a contraction mapping, so it converges to the unique fixed point.


\item Next, we concentrate out the linear parameters from the implied $\delta$ vectors. In particular, we estimate equation (5) via two stage least squares for each education by race group, using all our technical instruments (the wage shocks and their interactions with geographic and regulatory constraints). Note that in constructing the endogenous regressor $w_j^e - \zeta r_j$, we use the value of $\zeta$ conjectured in the outer loop.

\item At this point, we have conjectured values of all parameters, so we can form our moment conditions and take the empirical mean to form our GMM objective function. We run two-step GMM, in the first step using the identity matrix as weights in the first step, and in the second step using the inverse of the empirical residual matrix from the first step. 

\section*{Results}
Column 2 of Table 1 displays the results of our estimation procedure. We obtain precise estimates of all our true coefficient
\end{enumerate}

\section*{Tables and Figures}
\begin{table}[h]
\caption {True and Estimated Parameter Values}
\begin{center}
\label{tab:parameter_values} 
\begin{tabular}{l|rr} \textbf{Parameter} & \textbf{ True Value} & \textbf{Estimated Value} \\\hline \hline 
\multicolumn{2}{l}{Population parameters}  \\
\hline
$J$ & \paramsJ \\ 
$N_H^{White}$ & \paramsparamsparamsHWhiteN & \\   
$N_H^{Black}$ & \paramsparamsparamsHBlackN &\\   
$N_L^{White}$ & \paramsparamsparamsLWhiteN &\\   
$N_L^{Black}$ & \paramsparamsparamsLBlackN &\\   

\hline
\multicolumn{2}{l}{Labor supply elasticities}  \\
\hline
$\gamma_{HH}$ & \paramsgammaHH  & \paramsestgammaHH \\
$\gamma_{HL}$ & \paramsgammaHL & \paramsestgammaHL\\
$\gamma_{LL}$ & \paramsgammaLL  & \paramsestgammaLL \\ 
$\gamma_{LH}$ & \paramsgammaLH   & \paramsestgammaLH\\ 
\hline 

\multicolumn{2}{l}{Instrument first stages}  \\
\hline
$\alpha_{HH}$ & \paramsalphaHH  & \paramsestalphaHH\\
$\alpha_{HL}$ & \paramsalphaHL & \paramsestalphaHL\\
$\alpha_{LL}$ & \paramsalphaLL & \paramsestalphaLL\\ 
$\alpha_{LH}$ & \paramsalphaLH  & \paramsestalphaLH\\ 
\hline 

\multicolumn{2}{l}{Housing supply parameters}  \\
\hline

$\iota$ & \paramsiota  & \paramsestiota \\ 
$\phi$ & \paramsphi & \paramsestphi \\ 
$\phi_{geo}$ & \paramsphigeo & \paramsestphigeo \\ 
$\phi_{reg}$ & \paramsphireg & \paramsestphireg \\ 
\hline 
\multicolumn{2}{l}{Preference parameters (\{White, Black\})}  \\
\hline

$\beta_w^{White}$ & \paramsparamsbetawWhite  & \paramsestbetawWhite \\ 
$\beta_w^{Black}$ & \paramsparamsbetawBlack &  \paramsestbetawBlack \\ 
$\beta_a^{White}$ & \paramsparamsbetaaWhite & \paramsestbetaaWhite\\ 
$\beta_a^{Black}$ & \paramsparamsbetaaBlack & \paramsestbetaaBlack \\ 
$\beta_x^{White}$ & \paramsparamsbetaxWhite  & NA \\ 
$\beta_x^{Black}$ & \paramsparamsbetaxBlack &  NA \\ 

$\beta_{st}$ & \paramsbetast & \paramsestbetast\\ 
$\zeta$ & \paramszeta & XX \\ 
\hline 

\multicolumn{2}{l}{Distribution of observables }  \\
\hline


$\mu_{x\_reg}$ & \paramsparamsxregmu \\
$\sigma_{x\_reg}$ & \paramsparamsxregsigma \\
$\mu_{x\_geo}$ & \paramsparamsxgeomu \\
$\sigma_{x\_geo}$ & \paramsparamsxgeosigma \\

$\mu_{CC}$ & \paramsparamsCCmu \\
$\sigma_{CC}$ & \paramsparamsCCsigma \\ 
$\mu_{x}$ & \paramsparamsxmu \\
$\sigma_{x}$ & \paramsparamsxsigma \\
\hline
\multicolumn{2}{l}{Distribution of unobservables }  \\
\hline
$\mu_{\varepsilon_H}$ & \paramsparamsepsilonHmu \\ 
$\sigma_{\varepsilon_H}$ & \paramsparamsepsilonHsigma \\ 

$\mu_{\varepsilon_L}$ & \paramsparamsepsilonLmu \\ 
$\sigma_{\varepsilon_L}$ & \paramsparamsepsilonLsigma \\ 

$\mu_{\varepsilon_a}$ & \paramsparamsepsilonamu \\ 
$\sigma_{\varepsilon_a}$ & \paramsparamsepsilonasigma \\


$\mu_{Z_H}$ & \paramsparamsZHmu \\ 
$\sigma_{Z_H}$ & \paramsparamsZHsigma \\ 
$\mu_{Z_L}$ & \paramsparamsZLmu \\ 
$\sigma_{Z_H}$ & \paramsparamsZLsigma \\ 



\hline
\end{tabular}
\end{center}
\end{table}

%By BLP , there is a unique vector of mean utilities $\delta_{j}$ that imply the choice shares $H_{j}, L_{j}$ in the data.  From there, section D shows how $\Delta \delta_{jd}$, $\Delta p_{j}$, and the demographic variables form a residual term  that is orthogonal to our instrument $\Delta Z_{j}^x$ (and any interactions with local housing supply elasticities) at the true parameter values. Therefore, we use the moment conditions from that section to identify the nonlinear parameters in the inner loop. 


%We run our GMM via two-step, weighting with the identity matrix in the first step and the empirical variance-covariance matrix of our first step residuals in the second step. Standard errors are obtained using the usual GMM ``sandwich" formula.




%\begin{enumerate}
%\end{enumerate}

% \section{Extensions}

% \subsection{Complicating the Supply Side}

% The model above assumes percent competition on the supply side. We will consider more general frameworks of this model. Suppose that there is a firm that provides housing with production function,
% $$Y = F(L,B)$$
% where $L$ is land and is sold by oligopolistic landowners, and $B$ is construction costs, which are provided competitively.

% We can also think about the fact that firms produce houses, but households rent. In this case, 
% $$r_{j} = \sum_{s=t}^\infty \frac{\E[R_{js}]}{(1+r)^s} \approx \frac{R_{j}}{r-g}$$
% where $r$ is the discount rate and $g$ is the expected growth rate of rental prices. This is endogenous due to the endogenous supply of amenities.

% %% TODO: Think about transition dynamics, 


% %% We could also do something more complicated where we look at share of land available and the cost of constructing up vs horizontally etc, and create a more realistic marginal cost curve

% \subsection{A Model of Buyers and Renters}

% %% PROBABLY TOO COMPLICATED because we need to think about dynamics which is sad

% The model described earlier assumes that homeowners are absentee landlords and that the relationship between rental and sale prices is homogenous and dictated by an exogenous discount rate $\iota$, which is contrary to empirical evidence. We relax this assumption.

% We have several hypotheses as to why price-to-rent ratios may vary across cities:
% \begin{enumerate}
%     \item Variation in discount rate. E.g due to taxation. This could also be related to differences in risk across locations. For example, if a location is seen as more risky (e.g. higher probability of natural disaster/devaluation), it's price-to-rent ratio will be lower
%     \item Variation in expected growth rate. This could be related to expected future demand as well as well as the curvature of the supply function. For example, if the city is near peak capacity, any future demand shocks will raise future rents and prices more. This will raise the price-to-rent ratio
%     \item Variation in preference for homeownership. Suppose that there is an annual fixed cost to renting (e.g. probability of being kicked out, disutility of lack of control, etc), whereas there is a fixed cost to owning at transaction time (e.g. time costs of selling). Then, the utility of owning vs renting will vary depending on expected time of residence.
% \end{enumerate}
% Our goal is to incorporate these into the model to the degree that we can. 

% First, note that without endogenizing the local housing discount rate, $r_{j}$, we can still get variation in price to rent ratios by calculating expected increases in rents from endogenous changes in amenities, using the existing model.

% In the next section, we will attempt to endogenize this discount rate.

% \subsubsection{Housing Demand}
% Consider a household $i$ that at time $t$ chooses a city $j$ to live in and whether to purchase a home or rent. The household will earn wage $W_{ij}$ and will pay rent $R_{j}$ in every period, or it will pay $r_{j}$ once at $t=t_0$. The household has Cobb-Douglas preferences between a homogenous consumption good, $C$, and local amenities $A_{ij}$, and also has an idiosyncratic, time-invariant preference for each city, $\varepsilon_{ij}$. 

% The utility of renting for $T$ periods is,
% $$U^R_{it_0} = \sum_{t=t_0}^T\max_{j, C} \beta^t (1-\zeta) \log C + \zeta \log A_{ij} + \varepsilon_{ij}$$
% such that $C + R_{j} \leq W_{ij}$ for all $t\geq t_0$. We have normalized the price of the consumption good to 1.

% The utility of owning is,
% $$U^P_{it_0}= \sum_{t=t_0}^T\max_{j, C} \beta^t (1-\zeta) \log C + \zeta \log A_{ij} + \mu_{ij} + \varepsilon_{ij}$$
% such that $P_{t_0} + \sum_{t=t_0}^T C + \tau_j P = \sum_{t=t_0}^T \beta^t W_{ij}$. Observe that we have added a idiosyncratic preference shock $\mu_{ij}$ that represents $i$'s utility of buying vs renting in city $j$. The budget constraint for buyers says that the price of buying at $t_0$ plus total consumption and tax payments, with tax rate $\tau_j$, must be less than the present value of the household's wage. We assume that buyers purchase the house in full for simplicity.

% Substituting the budget constraint into the maximization problem, renters will choose
% $$j^*_R = \arg\max_j \sum_{t=t_0}^T \beta^t (1-\zeta) \log (W_{ij}-R_{j}) + \zeta \log A_{ij} + \varepsilon_{ij}$$
% and homeowners will choose
% $$j^*_P = \arg\max_j \sum_{t=t_0}^T \beta^t (1-\zeta) \log \left(W_{ij}-\frac{P_{t_0} + \sum_{t=t_0}^T \beta^t \tau_j P}{\sum_{t=t_0}^T \beta^t}\right) + \zeta \log A_{ij} + \varepsilon_{ij}$$


\end{document}