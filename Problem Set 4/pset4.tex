\documentclass{article}
\usepackage{graphicx}
\usepackage{amsmath}
\usepackage{bm}
\usepackage{mathptmx}
\usepackage[parfill]{parskip}

\usepackage{graphicx}
\usepackage{amsmath, amssymb, amsthm}
\usepackage{epsf}
\usepackage{bm}
\usepackage[parfill]{parskip}
\usepackage{mathptmx}
\usepackage{geometry}
\usepackage{fancyhdr}

\newcommand{\mc}[1]{\mathcal{#1}}
\newcommand{\mbf}[1]{\mathbf{#1}}
\def\E{\mathbb{E}}


\renewcommand{\thesection}{\Roman{section}}
\renewcommand{\thesubsection}{\Alph{subsection}}

\newcommand{\params_J}{100}
\newcommand{\params_J}{100}
\newcommand{\params_edu_types}{['H', 'L']}
\newcommand{\params_race_types}{['White', 'Black']}
\newcommand{\params_gamma_HH}{-0.2}
\newcommand{\params_gamma_HL}{0.5}
\newcommand{\params_gamma_LL}{-0.5}
\newcommand{\params_gamma_LH}{0.7}
\newcommand{\params_alpha_HH}{0.8}
\newcommand{\params_alpha_HL}{0.1}
\newcommand{\params_alpha_LH}{0.2}
\newcommand{\params_alpha_LL}{0.7}
\newcommand{\params_iota}{0.05}
\newcommand{\params_phi}{1}
\newcommand{\params_phi_geo}{0.3}
\newcommand{\params_phi_reg}{0.4}
\newcommand{\params_phi_a}{0.5}
\newcommand{\params_zeta}{0.8}
\newcommand{\params_beta_st}{3}
\newcommand{\params_beta_st}{3}
\newcommand{\params_beta_x}{{'White': 0.3, 'Black': 0.7}}
\newcommand{\params_beta_a}{{'White': 0.3, 'Black': 0.9}}
\newcommand{\params_beta_w}{{'White': 0.9, 'Black': 0.2}}
\newcommand{\params_H}{{'Black': {'N': 20}, 'White': {'N': 80}}}
\newcommand{\params_L}{{'Black': {'N': 180}, 'White': {'N': 20}}}
\newcommand{\params_Z_H}{{'mu': 0, 'sigma': 0.5}}
\newcommand{\params_Z_L}{{'mu': 0, 'sigma': 0.5}}
\newcommand{\params_x_geo}{{'mu': 0, 'sigma': 0.5}}
\newcommand{\params_x_reg}{{'mu': 0, 'sigma': 0.5}}
\newcommand{\params_CC}{{'mu': 0, 'sigma': 0.1}}
\newcommand{\params_x}{{'mu': 0, 'sigma': 0.1}}
\newcommand{\params_epsilon_H}{{'mu': 2, 'sigma': 0.1}}
\newcommand{\params_epsilon_L}{{'mu': 1, 'sigma': 0.1}}
\newcommand{\params_epsilon_a}{{'mu': 0, 'sigma': 0.1}}

\newcommand{\paramsestbetawWhite}{0.8995}
\newcommand{\paramsestbetaaWhite}{0.3019}
\newcommand{\paramsestbetawBlack}{0.1979}
\newcommand{\paramsestbetaaBlack}{0.9022}
\newcommand{\paramsestbetast}{3.0096}
\newcommand{\paramsestalphaHH}{0.8046}
\newcommand{\paramsestalphaHL}{0.0992}
\newcommand{\paramsestgammaHH}{-0.2117}
\newcommand{\paramsestgammaHL}{0.5241}
\newcommand{\paramsestalphaLH}{0.1916}
\newcommand{\paramsestalphaLL}{0.7019}
\newcommand{\paramsestgammaLH}{0.7264}
\newcommand{\paramsestgammaLL}{-0.5337}
\newcommand{\paramsestiota}{0.0501}
\newcommand{\paramsestphi}{0.9971}
\newcommand{\paramsestphigeo}{0.3007}
\newcommand{\paramsestphireg}{0.4012}
\newcommand{\paramsestphia}{0.4983}


\newcommand{\epszero}{\epsilon_{0t}}
\newcommand{\epsone}{\epsilon_{1t}}

\title{14.273 IO Problem Set 4}
\author{Benjamin Wittenbrink, Jack Kelly, and Verónica Bäcker-Peral}

\begin{document}

\maketitle
 \section*{Model}

This is a simple version of Ericson \& Pakes (1995). Firms discount the future by $\delta = 0.8$. In each period, firms face a quantity decision and an entry/exit decision. A state variable tracks whether firm $i$ is in the market ($s_{it} = 1$) or not ($s_{it} = 0$). There are four possible values for the state of the game at time $t$:

\[
s_t = (s_{1t}, s_{2t}) \in S = \{(0,0), (0,1), (1,0), (1,1)\}.
\]

If both firms are in the market ($s_t = (1,1)$), firms simultaneously make quantity choices $q_{it} \in \{0,1\}$ for duopoly profits
\[
\Pi^d_{it} = (1 + q_{it})A - q_{-it}B.
\]

If no more than one firm is in the market ($s_t \in \{(0,0), (0,1), (1,0)\}$), there is no quantity choice. The firm in the market makes monopoly profits
\[
\Pi^m_{it} = 2A,
\]
while profits for the firm(s) outside the market are
\[
\Pi^o_{it} = 0.
\]

Firms decide whether they want to be in the market for the next period or not. This choice is captured by $a_{it} \in \{0,1\}$, with $a_{it} = 1$ indicating that firm $i$ chooses to be in the market in period $t+1$. Incumbents that exit the market receive scrap values

\[
\varphi_{it} \sim \text{iid } U[0,1].
\]

Potential entrants that enter the market pay entry costs

\[
C + \psi_{it}, \quad \psi_{it} \sim \text{iid } U[0,1].
\]

Scrap values and entry costs are privately observed, but $C$ and the distributions of $\varphi_{it}$ and $\psi_{it}$ are common knowledge. Firms that exit the market are allowed to re-enter the market at a later point.

\newpage

\section*{Exercises}

\begin{enumerate}
    \item Write down the value function $V(s, \varepsilon_i)$ that describes the firm’s problem, for each of the four possible values of $s$ and for arbitrary $\varepsilon_i = (\varphi_i, \psi_i)$, as functions of
    \begin{itemize}
        \item (i) ex-ante value functions $V(s)$, and
        \item (ii) either four indifference cutoffs for $\varepsilon_{it}$\footnote{For example, if $s = (0,1)$, incumbent exits if scrap value is higher than some cutoff $\bar{\varphi}_{(0,1)/(1,0)}$ and entrant enters if the entry costs are lower than some cutoff $\bar{\psi}_{(0,1)/(1,0)}$.}
        or four choice probabilities $p_{-i}(s_i, s_{-i})$.
    \end{itemize}

    \textit{Hint:} Because current quantity decisions do not affect future states or profits, each firm chooses quantity as if they play a static duopoly game.

\begin{answer}
First note that the static duopoly game has the following solution: for $A>0$, $q_{it}=1$ is a strictly dominant strategy for each firm so there is a unique equilibrium of $q_{it}=1,q_{-it}=1$ yielding profits of $2A-B$ to each firm.


Then for each state $s$ we write the value function of the firm $i$ as the max of exiting/staying out of the market vs. entering/staying in the market plus the implied continuation value for that choice. For example, for $s = (0,0)$, the value of staying out is just the continuation value while the value of entering is the up front cost $-(C + \psi_{it})$ plus the continuation value. That is:
\[
  V(s,\varepsilon_i)
  = \underbrace{\pi_i(s)}_{\text{current payoff}} + \max_{a_i\in\{0,1\}}
    \Bigl\{
\underbrace{\varepsilon_i(a_i)}_{\text{scrap/entry payoff}}
+\delta \: \mathbb{E}\bigl[V(s'=(a_i,a_{-i}),\varepsilon')\mid s,a_i\bigr]\Bigr\}
\]
Thus, 
\begin{equation*}
    V(s \equiv (0,0),\varepsilon_i) = \max \{ \delta E[ V((0,a_{{-i}}),\varepsilon)\mid s],  -C - \psi_{it} +\delta E[ V((1,a_{{-i}}),\varepsilon] \mid s]\}.
\end{equation*}
Likewise for $s \in \{(0,1), (1,0), (1,1)\}$:

 \begin{align*}
 V(s \equiv (0,1),\varepsilon_i) & = \max \{  \delta E[ V((0,a_{{-i}}),\varepsilon) \mid s], -C- \psi_{it} + \delta E[ V((1,a_{{-i}}),\varepsilon \mid s] \} \\
 V(s \equiv (1,0),\varepsilon_i) & =2A +  \max \{ \varphi_{it}+ \delta E[ V((0,a_{{-i}}),\varepsilon) \mid s], \delta E[ V((1,a_{{-i}}),\varepsilon] \mid s]\} \\
 V(s \equiv (1,1),\varepsilon_i) & = 2A-B +\max \{\varphi_{it} + \delta E[ V((0,a_{{-i}}),\varepsilon) \mid s], \delta E[ V((1,a_{{-i}}),\varepsilon \mid s] \} 
\end{align*}

Let $\tilde{V}(s_1,s_2) = E[V((s_1,s_2), \varepsilon)]$ and 
\begin{align*}
    p_{i}(s_1,s_2) &= Pr(a_{i} = 1 \mid s = (s_1,s_2))\\
    p_{-i}(s_1,s_2) &= Pr(a_{-i} = 1 \mid s = (s_1,s_2)),
\end{align*}
Observe that since the problem is symmetric we have that $p_i(s_1,s_2) = p_{-i}(s_2,s_1)$. Then, note that 
\begin{align*}
        E[V((s_1',a_{-i}),\epsilon)|(s_1,s_2)] &= p_{-i}(s_1,s_2) E[V((s_1',1),\epsilon)] + (1- p_{-i}(s_1,s_2))E[V((s_1',0),\epsilon] \\
        &= p_{-i}(s_1,s_2) \tilde{V}(s_1',1)+(1- p_{-i}(s_1,s_2))\tilde{V}(s_1',0).
\end{align*}
Therefore, our value function simplifies to, 
\[
  V(s,\varepsilon_i)
  = \underbrace{\pi_i(s)}_{\text{current payoff}} + \max_{a_i\in\{0,1\}}
    \Bigl\{
      \underbrace{\varepsilon_i(a_i)}_{\text{scrap/entry}}
      +\delta
      \bigl[p_{-i}(s)\tilde{V}(a_i,1)
      +(1-p_{-i}(s))\,\tilde{V}(a_i,0)\bigr]
    \Bigr\}
\]
where current-period payoff $\pi_i(s)$ is given by
\[
\pi_i(s) =
\begin{cases}
  0 & \text{if } s = (0,0), \\
  2A & \text{if } s \in \{(1,0), (0,1)\}, \\
  2A - B & \text{if } s = (1,1).
\end{cases}
\]
and the additional payoff $\varepsilon_i(a_i)$ from the entry or exit decision is
\[
\varepsilon_i(a_i) =
\begin{cases}
  - (C + \psi_{it}) & \text{if } s_i = 0 \text{ and } a_i = 1 \quad \text{(entry)}, \\
  \varphi_{it} & \text{if } s_i = 1 \text{ and } a_i = 0 \quad \text{(exit)}, \\
  0 & \text{otherwise}.
\end{cases}
\]



\end{answer}

\item Let $(A, B, C) = (0.3, 0.6, 0.15)$. Solve for the game’s equilibrium by finding the roots to a system of non-linear equations.

\textit{Hint:} You should have eight equations in eight unknowns. The eight equations include four Bellman equations and four indifference/choice probability equations; the eight unknowns include four $V(s)$ and four cut-offs/choice probabilities.
   
    If solving for cut-offs, derive choice probabilities in the last step.

\begin{answer}

The eight unknowns will be the four integrated value functions $\tilde{V}(s)$ and four choice probabilities $p_{-i}(s)$ for $s \in \{(0,0), (0,1), (1,0), (1,1)\}$.

To proceed, suppose first that firm 1 is outside the market. Then there will be a cutoff $\bar{\psi}(s_2)$ such that firm 1 is indifferent between entering or staying outside. Setting the two arguments of the RHS of the Bellman equation equal to each other and solving for $\bar{\psi}$ we get that,
\[
\bar{\psi}(s_2) = \delta p_{-i}(0,s_2)(\tilde{V}(1,1)-\tilde{V}(0,1)) +  \delta (1- p_{-i}(0,s_2))(\tilde{V}(1,0)-\tilde{V}(0,0)) - C
\]

Similarly, when firm 1 is inside the market, there is a cutoff $\bar{\varphi}(s_2)$ such that firm 1 is indifferent between staying insider or exiting. We get,
\[
\bar{\varphi}(s_2) = \delta p_{-i}(1,s_2)(\tilde{V}(1,1)-\tilde{V}(0,1)) + \delta (1-p_{-i}(1,s_2))(\tilde{V}(1,0)-\tilde{V}(0,0))
\] 

Therefore, using the fact that the  probabilities are symmetric for firm 1 and 2, and that $\psi$ and $\varphi$ are uniform on [0,1],
\begin{align*}
    p_{-i}(s_1,0) &= Pr(\psi < \bar{\psi}(s_1)) \\
    &= \bar{\psi}(s_1)
\end{align*}
and
\begin{align*}
    p_{-i}(s_1,1) &= Pr(\varphi < \bar{\varphi}(s_1)) \\
    &= \bar{\varphi}(s_1)
\end{align*}
which gives us four equations for each choice probability.

Moreover, we can solve for the expected utility when firm 1 is currently outside the market,
\begin{align*}
    \tilde{V}(0,s_2) &= E[V((0,s_2),\varepsilon] \\
    &= p_{-i}(0,s_2)p_{-i}(s_2,0)\Bigr[-C + E[\psi|\psi<\bar{\psi}(s_2)] +\delta\tilde{V}(1,1)\Bigr]\\
    &\quad + p_{-i}(0,s_2)(1-p_{-i}(s_2,0))\Bigr[\delta\tilde{V}(0,1)\Bigr]\\
    &\quad + (1-p_{-i}(0,s_2))p_{-i}(s_2,0)\Bigr[-C + E[\psi|\psi<\bar{\psi}(s_2)] +\delta\tilde{V}(1,0)\Bigr]\\
    &\quad + (1-p_{-i}(0,s_2))(1-p_{-i}(s_2,0))\Bigr[\delta\tilde{V}(0,0)\Bigr] \\
    &= p_{-i}(0,s_2)p_{-i}(s_2,0)\Bigr[-C + \frac{\bar{\psi}(s_2)}{2} +\delta\tilde{V}(1,1)\Bigr]\\
    &\quad + p_{-i}(0,s_2)(1-p_{-i}(s_2,0))\Bigr[\delta\tilde{V}(0,1)\Bigr]\\
    &\quad + (1-p_{-i}(0,s_2))p_{-i}(s_2,0)\Bigr[-C + \frac{\bar{\psi}(s_2)}{2} +\delta\tilde{V}(1,0)\Bigr]\\
    &\quad + (1-p_{-i}(0,s_2))(1-p_{-i}(s_2,0))\Bigr[\delta\tilde{V}(0,0)\Bigr]
\end{align*}
and the expected utility when firm 1 is currently in the market,
\begin{align*}
    \tilde{V}(1,s_2) &= E[V((1,s_2),\varepsilon] \\
    &= \pi_i(1,s_2) \\
    &\quad + p_{-i}(1,s_2)p_{-i}(s_2,1)\Bigr[\delta\tilde{V}(1,1)\Bigr]\\
    &\quad + p_{-i}(1,s_2)(1-p_{-i}(s_2,1))\Bigr[E[\varphi | \varphi > \tilde{\varphi}] + \delta\tilde{V}(0,1)\Bigr]\\
    &\quad + (1-p_{-i}(1,s_2))p_{-i}(s_2,1)\Bigr[\delta\tilde{V}(1,0)\Bigr]\\
    &\quad + (1-p_{-i}(1,s_2))(1-p_{-i}(s_2,1))\Bigr[E[\varphi | \varphi > \tilde{\varphi}] + \delta\tilde{V}(0,0)\Bigr] \\
    &= \pi_i(1,s_2) \\
    &\quad + p_{-i}(1,s_2)p_{-i}(s_2,1)\Bigr[\delta\tilde{V}(1,1)\Bigr]\\
    &\quad + p_{-i}(1,s_2)(1-p_{-i}(s_2,1))\Bigr[\frac{1+\bar{\varphi}(s_2)}{2}+\delta\tilde{V}(0,1)\Bigr]\\
    &\quad + (1-p_{-i}(1,s_2))p_{-i}(s_2,1)\Bigr[\delta\tilde{V}(1,0)\Bigr]\\
    &\quad + (1-p_{-i}(1,s_2))(1-p_{-i}(s_2,1))\Bigr[\frac{1+\bar{\varphi}(s_2)}{2}+\delta\tilde{V}(0,0)\Bigr]
\end{align*}
for $s_2\in\{0,1\}$.

Therefore, we end up with 8 equations and 8 unknowns. We rewrite the 8 equations below for convenience,
\begin{align}
    p_{00} &= -C + \delta \left[p_{00}\Bigr(\tilde{V}_{11}-\tilde{V}_{01}\Bigr) + q_{00}\Bigr(\tilde{V}_{10}-\tilde{V}_{00}\Bigr)\right] \\
    p_{10} &= -C + \delta \left[p_{01}\Bigr(\tilde{V}_{11}-\tilde{V}_{01}\Bigr) + q_{01}\Bigr(\tilde{V}_{10}-\tilde{V}_{00}\Bigr)\right] \\
    p_{01} &= \delta \left[p_{10}\Bigr(\tilde{V}_{11}-\tilde{V}_{01}\Bigr) + q_{10}\Bigr(\tilde{V}_{10}-\tilde{V}_{00}\Bigr)\right] \\
    p_{11} &= \delta \left[p_{11}\Bigr(\tilde{V}_{11}-\tilde{V}_{01}\Bigr) + q_{11}\Bigr(\tilde{V}_{10}-\tilde{V}_{00}\Bigr)\right] \\
    \tilde{V}_{00} &= -p_{00}\Bigr[C + \frac{p_{00}}{2} \Bigr] + \delta\Bigr(p_{00}(p_{00}\tilde{V}_{11} + q_{00}\tilde{V}_{10}) + q_{00}(p_{00}\tilde{V}_{01} + q_{00}\tilde{V}_{00})\Bigr) \\
    \tilde{V}_{01} &= -p_{10}\Bigr[C + \frac{p_{10}}{2} \Bigr] + \delta\Bigr(p_{10}(p_{01}\tilde{V}_{11} + q_{01}\tilde{V}_{10}) + q_{10}(p_{01}\tilde{V}_{01} + q_{01}\tilde{V}_{00})\Bigr) \\
    \tilde{V}_{10} &=\pi_{10} + \frac{q_{01}(1+p_{01})}{2}   + \delta\Bigr(p_{01}(p_{10}\tilde{V}_{11} + q_{10}\tilde{V}_{10}) + q_{01}(p_{10}\tilde{V}_{01} + q_{10}\tilde{V}_{00}) \Bigr) \\
    \tilde{V}_{11}&=\pi_{11} + \frac{q_{11}(1+p_{11})}{2}  + \delta\Bigr(p_{11}(p_{11}\tilde{V}_{11} + q_{11}\tilde{V}_{10})+q_{11}(p_{11}\tilde{V}_{01} + q_{11}\tilde{V}_{00})\Bigr)
\end{align}
where $\tilde{V}_{ij}\equiv\tilde{V}(i,j)$, $\pi_{ij}\equiv\pi_{i}(i,j)$ $p_{ij}\equiv p_{-i}(i,j)$ and $q_{ij}\equiv 1-p_{-i}(i,j)$


We solve the system of equations numerically for the conjectured parameters via a root finder. In particular, we conjecture values of our 8 unknowns (the LHSs of equations 1-8) and use those equations to get updated values.

We find the following solution:
\begin{figure}[h]
\centering
\caption{Solution to Dynamic Game}
\begin{tabular}{l|r}  
Parameter & Solution \\ \hline
$V_{00}$ & 0.65 \\
$V_{01}$ & 0.59 \\
$V_{10}$ & 1.89 \\
$V_{11}$ & 1.21 \\
$p_{00}$ & 0.56 \\
$p_{01}$ & 0.76 \\
$p_{10}$ & 0.46 \\
$p_{11}$ & 0.66 \\
 \end{tabular}
\end{figure}

This solution is intuitive: the firms prefer to be acting as monopolists (1,0) to duopolists (1,1), and both are better than not having entered at all, but the continuation value is more appealing when the other firm has also not entered $V_{00}>V_{01}$ since the probability they enter in the subsequent period is lower. 
\end{answer}



    \item Generate data from your solution in (2), where you derived choice probabilities conditional on the four possible states. Read \textit{“Asymptotic Least Squares Estimators for Dynamic Games”} by Pesendorfer and Schmidt-Dengler and implement their estimator.

    \textit{Hint:} First express $V$ in terms of parameters $\theta$, known primitives, and estimated choice probabilities $\hat{p}$. \footnote{You can use forward simulation, although in this case, you have a nice analytical closed form for $V(\hat{p}, \theta)$.}

    Once you have $V(\hat{p}, \theta)$, you can obtain guesses for the choice-specific value functions and choice probabilities $\tilde{p}(\hat{p}, \theta)$, build a least square objective function over deviations from observed choice probabilities $\hat{p} - \tilde{p}(\hat{p}, \theta)$, and search over unknown parameters $\theta = (A,B,C)$ that minimize these deviations.

\begin{answer}

First, we simulate the data. We initialize the state at $(0,0)$. Then, we simulate state $s_t$ by drawing $\phi_t, \psi_t$ for each firm, and choose whether to enter or exit in the following period using the solutions to the system of equations above, the $\phi,\psi$ draws, and the existing state. We simulate this for 100,000 time periods. 

Next we implement the estimator of Pesendorfer and Schmidt-Dengler. This estimator proceeds in two steps 

\begin{enumerate}
\item Estimate $p_{i,j}$ using empirical frequency. In particular, let $a_{i}^t$ denote the entry decision of firm $i$ in period $t$, equal to 1 iff they enter, and let $s_i^t$ denote the state \textit{relative} to agent $i$: that is, if the agent is $1$ and the state is $(1,0)$, $s_i^t = (1,0)$, while if the agent is $2$ and the state is $(0,1)$, $s_i^t = (1,0)$. We estimate $$\hat{p}(s) = \frac{ \sum_{i\in\{0,1\}}\sum_t I\{a_i^t=1,s_i^t=s\}}{\sum_{a\in\{0,1\}} \sum_{i\in\{0,1\}}\sum_t I\{a_i^t=1,s^t=s\}}$$ for each of the four states $s$

\item For conjectured values of $A,B,C$, we solve the model as in part (2). This yields model-implied values of $\tilde p(s)$ for each of the four states. We form the moment condition $$ M=E[\tilde{p}(s)-\hat{p}(s)]=0,\forall s$$

and choose $\tilde{A},\tilde{B},\tilde{C}$ to minimize the GMM objective function $$\hat{M}^\prime W \hat{M},$$, where $\hat{G}$ is the sample analogue of $M$. We do our estimation via two-step: in the first step, $W=I_{4}$. In the second step, since entry is i.i.d. Bernoulli conditional on the state, we simply have $\hat W = \frac{1}{4}\hat{\Sigma}^{-1}$ where $\hat{\Sigma}$ has $  \frac{\hat{p}(s)(1-\hat{p}(s))}{n_s}$ in the $s,s$th component and 0 otherwise, for $n_s$ the number of observations with state $s$. We also compute standard errors using the method in Pesendorfer and Schmidt-Dengler.


\begin{figure}[h]
\centering
\caption{Estimated Parameters,  Pesendorfer and Schmidt-Dengler Estimator}
\begin{tabular}{l|rr r }  
& A & B & C \\
\hline

True Value  & 0.300 &0.600& 0.150 \\ 
Estimated Value& 0.294 &0.578& 0.152 \\ 
Standard Error & 0.009 &0.031&  0.007
 \end{tabular}
\end{figure}

\end{enumerate}
\end{answer}
    
\end{enumerate}


\end{document}