\documentclass{article}
\usepackage{graphicx}
\usepackage{amsmath}
\usepackage{bm}
\usepackage{mathptmx}
\usepackage[parfill]{parskip}

\usepackage{graphicx}
\usepackage{amsmath, amssymb, amsthm}
\usepackage{epsf}
\usepackage{bm}
\usepackage[parfill]{parskip}
\usepackage{mathptmx}
\usepackage{geometry}
\usepackage{fancyhdr}

\newcommand{\mc}[1]{\mathcal{#1}}
\newcommand{\mbf}[1]{\mathbf{#1}}
\def\E{\mathbb{E}}


\renewcommand{\thesection}{\Roman{section}}
\renewcommand{\thesubsection}{\Alph{subsection}}

\newcommand{\params_J}{50}
\newcommand{\params_edu_types}{['H', 'L']}
\newcommand{\params_race_types}{['White', 'Black']}
\newcommand{\params_gamma_HH}{-0.2}
\newcommand{\params_gamma_HL}{0.5}
\newcommand{\params_gamma_LL}{-0.5}
\newcommand{\params_gamma_LH}{0.7}
\newcommand{\params_alpha_HH}{0.8}
\newcommand{\params_alpha_HL}{0.1}
\newcommand{\params_alpha_LH}{0.2}
\newcommand{\params_alpha_LL}{0.7}
\newcommand{\params_iota}{0.05}
\newcommand{\params_phi}{1}
\newcommand{\params_phi_geo}{0.3}
\newcommand{\params_phi_reg}{0.4}
\newcommand{\params_phi_a}{0.5}
\newcommand{\params_zeta}{0.8}
\newcommand{\params_beta_st}{5}
\newcommand{\params_beta_x}{{'White': 0.3, 'Black': 0.7}}
\newcommand{\params_beta_a}{{'White': 0.3, 'Black': 0.9}}
\newcommand{\params_beta_w}{{'White': 0.9, 'Black': 0.2}}
\newcommand{\params_H}{{'Black': {'N': 20}, 'White': {'N': 80}}}
\newcommand{\params_L}{{'Black': {'N': 180}, 'White': {'N': 20}}}
\newcommand{\params_Z_H}{{'mu': 0, 'sigma': 0.5}}
\newcommand{\params_Z_L}{{'mu': 0, 'sigma': 0.5}}
\newcommand{\params_x_geo}{{'mu': 0, 'sigma': 0.5}}
\newcommand{\params_x_reg}{{'mu': 0, 'sigma': 0.5}}
\newcommand{\params_CC}{{'mu': 0, 'sigma': 0.01}}
\newcommand{\params_x}{{'mu': 0, 'sigma': 0.01}}
\newcommand{\params_epsilon_H}{{'mu': 2, 'sigma': 0.01}}
\newcommand{\params_epsilon_L}{{'mu': 1, 'sigma': 0.01}}
\newcommand{\params_epsilon_a}{{'mu': 0, 'sigma': 0.01}}

\newcommand{\paramsestbetawWhite}{0.8995}
\newcommand{\paramsestbetaaWhite}{0.3019}
\newcommand{\paramsestbetawBlack}{0.1979}
\newcommand{\paramsestbetaaBlack}{0.9022}
\newcommand{\paramsestbetast}{3.0096}
\newcommand{\paramsestalphaHH}{0.8046}
\newcommand{\paramsestalphaHL}{0.0992}
\newcommand{\paramsestgammaHH}{-0.2117}
\newcommand{\paramsestgammaHL}{0.5241}
\newcommand{\paramsestalphaLH}{0.1916}
\newcommand{\paramsestalphaLL}{0.7019}
\newcommand{\paramsestgammaLH}{0.7264}
\newcommand{\paramsestgammaLL}{-0.5337}
\newcommand{\paramsestiota}{0.0501}
\newcommand{\paramsestphi}{0.9971}
\newcommand{\paramsestphigeo}{0.3007}
\newcommand{\paramsestphireg}{0.4012}
\newcommand{\paramsestphia}{0.4983}


\title{14.273 IO Problem Set 2}
\author{Benjamin Wittenbrink, Jack Kelly, and Veronicá Bäcker-Peral}

\begin{document}

\maketitle

\section{Production Function Estimation}

This problem uses data from the NBER Working Paper Grilliches and Mairesse (1995) (henceforth, GM), and builds on previous problem sets by Aviv Nevo, Allan Collard-
Wexler, and Jan DeLoecker. The dataset (\texttt{gmData.csv}) contains nine variables: 
\begin{itemize}
\item \texttt{index}: Firm ID
\item \texttt{sic3}: 3-digit SIC code
\item \texttt{yr}: Year
\item \texttt{ldsal}: Log deflated sales
\item \texttt{lemp}: Log employment
\item \texttt{ldnpt}: Log deflated capital 
\item \texttt{ldrst}: Log deflated R\&D capital
\item \texttt{ldrnd}: Log deflated R\&D
\item \texttt{ldinv}: Log deflated investment
\end{itemize}
For additional details on the data, see Hall (1990). 

We are interested in estimating the following production function:
\begin{equation}\label{p1_prod_fn}
ldsal_{it} = \beta_1lemp_{it} + \beta_2ldnpt_{it} + \beta_3ldrst_{it} + d_t + d_{t, sic357} + \omega_{it} + \varepsilon_{it},
\end{equation}
where $d_t$ denotes year fixed effects and $ d_{t, sic357}$ denotes year fixed effects for industry 357. 

Answer the following questions:

\begin{enumerate}
\item \textbf{Reproducing GM Results:}

\begin{enumerate}

\item Reproduce Columns 1–4 of Table 3 in GM by estimating different versions of
Equation (\ref{p1_prod_fn}).


\item Compare the estimates in Column 3 and Column 1. What insights do you
draw from the differences?




\item Compare these results with Column 1 of Table VI in Olley and Pakes (1996).
What do you learn from the comparison?

\end{enumerate}

\item \textbf{Dynamic Panel Model:}
Following the dynamic panel model in Ackberg, Caves, and Frazer (2015) (henceforth, ACF) in Section 4.3.3:

\begin{enumerate}
\item Derive the $\rho$–dfferenced version of Equation (\ref{p1_prod_fn}).



\item State the analog of ACF’s assumptions in this context. 



\item Write the corresponding version of ACF’s Equation (34).





\item List the set of valid instruments based on your stated assumptions.




\item Estimate the dynamic panel model using GMM and compare the results with
those from Equation (\ref{p1_prod_fn}).


\end{enumerate}


\item \textbf{ACF: First Stage Moments:}

\begin{enumerate}
\item Write down the first-stage ACF moment condition.



\item Estimate the first-stage ACF regression using \texttt{ldinv} as the measure of investment.



\end{enumerate}

\item \textbf{ACF: Second Stage Moments:}

\begin{enumerate}
\item Write down the second-stage ACF moment condition.



\item Estimate the second-stage ACF model to recover $\beta_1, \beta_2,$ and $\beta_3$ under the assumption $\omega_{it} = \rho \omega_{i t-1} + \mu + \xi_{it}.$


\item Compare these results with those from both Equation (\ref{p1_prod_fn}) and your previous
estimation.


\item Contrast your findings with Column 5 of GM.

\end{enumerate}


\item \textbf{ACF Moments with Survival Control:}
Redo the second-stage ACF moments incorporating a survival control in the evolution of $\omega$ (i.e,. where survival depends on capital and year). Compare your new estimates with those obtained in part (4) and with Column 6 of GM.



\item \textbf{Mark-Up Calculation:}
Calculate mark-ups using labor as a variable input (consult De Loecker and Warzynski (2012) for reference). Note that \texttt{lemp} represents the logarithm of the quantity
of employed workers (in thousands). To do this:
\begin{enumerate}
\item Merge the industry-level wage data from \texttt{sic5811.csv}.


\item Plot both the share-weighted and unweighted time series of mark-ups. 


\item Comment on your findings. 
\end{enumerate}


\end{enumerate}




\end{document}