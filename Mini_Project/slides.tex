\documentclass{beamer}

\AtBeginSection[]{
  \begin{frame}
  \vfill
  \centering
  \begin{beamercolorbox}[sep=8pt,center,shadow=true,rounded=true]{title}
    \usebeamerfont{title}\insertsectionhead\par%
  \end{beamercolorbox}
  \vfill
  \end{frame}
}



\usepackage{graphicx}
\usepackage{amsmath, amssymb, amsthm}
\usepackage{epsf}
\usepackage{bm}
\usepackage[parfill]{parskip}
\usepackage{mathptmx}
\usepackage{geometry}
\usepackage{fancyhdr}

\newcommand{\mc}[1]{\mathcal{#1}}
\newcommand{\mbf}[1]{\mathbf{#1}}
\def\E{\mathbb{E}}


\renewcommand{\thesection}{\Roman{section}}
\renewcommand{\thesubsection}{\Alph{subsection}}

\newcommand{\params_J}{50}
\newcommand{\params_edu_types}{['H', 'L']}
\newcommand{\params_race_types}{['White', 'Black']}
\newcommand{\params_gamma_HH}{-0.2}
\newcommand{\params_gamma_HL}{0.5}
\newcommand{\params_gamma_LL}{-0.5}
\newcommand{\params_gamma_LH}{0.7}
\newcommand{\params_alpha_HH}{0.8}
\newcommand{\params_alpha_HL}{0.1}
\newcommand{\params_alpha_LH}{0.2}
\newcommand{\params_alpha_LL}{0.7}
\newcommand{\params_iota}{0.05}
\newcommand{\params_phi}{1}
\newcommand{\params_phi_geo}{0.3}
\newcommand{\params_phi_reg}{0.4}
\newcommand{\params_phi_a}{0.5}
\newcommand{\params_zeta}{0.8}
\newcommand{\params_beta_st}{5}
\newcommand{\params_beta_x}{{'White': 0.3, 'Black': 0.7}}
\newcommand{\params_beta_a}{{'White': 0.3, 'Black': 0.9}}
\newcommand{\params_beta_w}{{'White': 0.9, 'Black': 0.2}}
\newcommand{\params_H}{{'Black': {'N': 20}, 'White': {'N': 80}}}
\newcommand{\params_L}{{'Black': {'N': 180}, 'White': {'N': 20}}}
\newcommand{\params_Z_H}{{'mu': 0, 'sigma': 0.5}}
\newcommand{\params_Z_L}{{'mu': 0, 'sigma': 0.5}}
\newcommand{\params_x_geo}{{'mu': 0, 'sigma': 0.5}}
\newcommand{\params_x_reg}{{'mu': 0, 'sigma': 0.5}}
\newcommand{\params_CC}{{'mu': 0, 'sigma': 0.01}}
\newcommand{\params_x}{{'mu': 0, 'sigma': 0.01}}
\newcommand{\params_epsilon_H}{{'mu': 2, 'sigma': 0.01}}
\newcommand{\params_epsilon_L}{{'mu': 1, 'sigma': 0.01}}
\newcommand{\params_epsilon_a}{{'mu': 0, 'sigma': 0.01}}

\newcommand{\paramsestbetawWhite}{0.8995}
\newcommand{\paramsestbetaaWhite}{0.3019}
\newcommand{\paramsestbetawBlack}{0.1979}
\newcommand{\paramsestbetaaBlack}{0.9022}
\newcommand{\paramsestbetast}{3.0096}
\newcommand{\paramsestalphaHH}{0.8046}
\newcommand{\paramsestalphaHL}{0.0992}
\newcommand{\paramsestgammaHH}{-0.2117}
\newcommand{\paramsestgammaHL}{0.5241}
\newcommand{\paramsestalphaLH}{0.1916}
\newcommand{\paramsestalphaLL}{0.7019}
\newcommand{\paramsestgammaLH}{0.7264}
\newcommand{\paramsestgammaLL}{-0.5337}
\newcommand{\paramsestiota}{0.0501}
\newcommand{\paramsestphi}{0.9971}
\newcommand{\paramsestphigeo}{0.3007}
\newcommand{\paramsestphireg}{0.4012}
\newcommand{\paramsestphia}{0.4983}


\pagestyle{fancy}
\fancyfoot[C]{\thepage} % page number in the center of the footer
\renewcommand{\headrulewidth}{0pt} % remove header rule if desired
\renewcommand{\footrulewidth}{0pt} % remove footer rule if desired
% read in parameters


\title{14.273 IO Mini-Project}
\author{Benjamin Wittenbrink, Jack Kelly, and Verónica Bäcker-Peral}

\begin{document}

\maketitle

\section{Motivation}

\begin{frame}{Overview}
\begin{itemize}
\item The paper: ``The Determinants and Welfare Implications of US Workers' Diverging Location Choices by Skill: 1980-2000" (Diamond \textit{AER} 2016)
\item Big picture: structural model of labor demand, migration, housing prices, and amenities to estimate why high-skill workers are increasingly located in high-wage cities (``The Great Divergence") and estimate its welfare implications
    \begin{itemize}
    \item H0: High wage cities are high price cities $\implies$ wage differences overstate welfare differences 
    \item H1: High wage cities are high amenity cities even conditional on housing prices $\implies$ wage differences understate welfare differences
    \end{itemize}

    \item Our main simplifications:  one exogenous and one endogenous amenity, simple scalar instrument vs. shift-share, one time period 
    %(reinterpret everything in first differences)

    \item Tools from the class: BLP-like estimation, GMM.

\end{itemize}
\end{frame}

\section{Model}

% \begin{frame}{Notation}
% \begin{itemize}
% \item City $j$
% \item High- and low-skill population $H_j$ and $L_j$
% \item 
% \end{itemize}

    
% \end{frame}


 \begin{frame}{Labor Demand}

\begin{itemize}
\item For city $j$ and high- and low-skill populations $H_j$ and $L_j$, follow the paper in specifying wages as 
 \begin{align}
     w_{j}^H &= \gamma_{HH} \log H_{j} + \gamma_{HL}  \log L_{j} +  \varepsilon_{j}^H \label{eq_wage_h_aprx} 
    \\ 
     w_{j}^L &= \gamma_{LL}  \log L_{j} + \gamma_{LH}   \log H_{j} +  \varepsilon_{j}^L \label{eq_wage_l_aprx} 
\end{align}

\item Reduced-form approximation that captures both classical labor demand effects and endogenous productivity from (relative) supply of different types
\end{itemize}

\end{frame}

%\begin{equation}
%\%end{equation}
%where $R_{jt}$ are local rents and $W_{jt}^{S}$ are skill  specific wages.

 \begin{frame}{Labor Supply }
 
 No choice of how much labor to supply, just where to locate. Location/migration model $\approx$ BLP
\begin{itemize}
\item Characteristics:  (skill-specific) log wages, log rents, exogenous amenity $x_{j}^a$ (unobserved -- this is  $\xi_{j}$ in BLP) , endogenous amenity $a_j$ (observed), and indicator for individual's state of birth $st_i$.
\begin{itemize}
\item $st_i$ observed, but varies even conditional on skill type: $\implies $ can't use logit directly, BLP-like estimation.
\end{itemize}

\item Following the paper, coefficients vary with race indicator $z$, whose distribution differs for $H$ and $L$, yielding mean utility: 
$$\delta_{jt}^z = (w_{j} - \zeta r_{j})\beta^w z + a_{j}\beta^a z + x_{j}^A\beta^xz$$
and education-specific pop. shares of 
\begin{equation}
S_{j} = \sum_{i\in\mc{X}} \frac{\exp(\delta_{j}^{z_i} + x_j^{st}st_i\beta^{st}z_i)}{\sum_{k} \exp(\delta_{jt}^{z_i} + x_j^{st}st_i\beta^{st}z_i)}, X \in {H,L} 
\end{equation}

\end{itemize}

\end{frame}

\begin{frame}{Local Good Demand}
Indirect utility function in previous slide is microfounded with Cobb-Douglas utility over a national good and a local housing good. Letting log wages relative to national good be $w^S_{jt}$ for each skill type and log rents $r_{jt}$, we then have local housing demand of 
\begin{equation}
    HD_{ij} = \zeta \exp \left(\frac{w^X_{j}}{r_{j}}\right), X \in \{H,L\}
\end{equation}

Why does this matter? In BLP don't care how money is spent on cars versus other goods -- but here we are endogenizing the price of other goods (housing) too.
\end{frame}

\begin{frame}{Housing and Amenity Supply}
\begin{itemize}
\item Assume
\begin{equation}\label{eq: housing supply}
    r_{j}  = \log \iota + \log CC_{j} + \varphi_j \log HD_{j}.
\end{equation}

where $\iota$ is housing interest rate, $CC_{j}$ are city-specific construction costs (\textbf{both unobserved}), and $\log HD_{j}$ is log local good consumption as before. We assume that $$\varphi_j  = \varphi +\varphi^{geo} \exp(x_j^{geo}) + \gamma^{reg}\exp(x_j^{reg}) $$
\item Key interaction effect: demand for local goods raises land costs and thereby rental rates
\item Endogenous amenities given by \begin{equation} a_{j}= \gamma^a \ln \left(\frac{H_j}{L_j}\right) + \varepsilon^a_j\end{equation}
\item Key interaction effect: ratio of high to low skill types affects level of amenity (e.g., nicer schools or parks).

\end{itemize}

\end{frame}

\begin{frame}{Equilibrium}
Equilibrium given by 
market clearing in 
\begin{itemize}
\item Skill-specific labor demand and supply/population: (1) and (2) and (3) hold simultaneously 
\item Housing supply implied by (5) equals housing demand, given by (4) weighted by skill types: $$ HD_{j}^* = L_{j}^* \frac{\zeta \exp(w_{j}^{L^*})}{\exp(r_{j}^*)} + H_{j}^* \frac{\zeta \exp(w_{j}^{H^*})}{\exp(r_{j}^*)}$$

\item Amenity supply equals amenity ``demand", given by (5) and (3) holding simultaneously
 
\end{itemize}
\end{frame}
\section{Identification and Estimation}

% \begin{frame}{Overview}
% Two identification challenges
% \begin{itemize}
%     \item Unobserved quality of locations -- modeled as unobserved exogenous amenity $x_{jt}^a$ -- handle using BLP approach
%     \item Simultaneity of supply and demand -- need an instrument
%     \end{itemize}
% Diamond uses Bartik-style instruments; we will instead posit an instrument $z_j^S$ varying at the city by skill level that affects wages independently of labor supply. 

% In particular, for $X \in \{H, L\}$ we assume 
%  $$ w_{j}^X = \gamma_{HH}  \ln H_{j} + \gamma_{HL} \ln L_{j}+\gamma_{BSH} Z_{j}^H  + \gamma_{BSL} Z_{j}^L + \tilde{\varepsilon}_{jt}^X
%  $$

% where $Z^H_j, Z^L_j \perp \tilde{\varepsilon}_{j}^X$. 
% This directly implies our first moment conditions: at the true parameter values,  
% $$E[ \tilde\varepsilon_{jt}^X Z_{jt}^Y]=0, \forall X,Y \in \{H,L\} \times \{H,L\}.$$ 

%  \end{frame}


 \begin{frame}{Overview}
Two identification challenges
\begin{itemize}
    \item Unobserved quality of locations -- modeled as unobserved exogenous amenity $x_{jt}^a$ -- handle using BLP approach
    \item Simultaneity of supply and demand -- need an instrument
    \end{itemize}
Diamond uses Bartik-style instruments; we will instead posit an instrument $z_j^S$ varying at the city by skill level that affects wages independenly of labor supply. 

In particular, for $ X \in\{H,L\}$ we assume 
 $$ w_{j}^X = \gamma_{HH}  \ln H_{j} + \gamma_{HL} \ln L_{j}+\gamma_{BSH} Z_{j}^H  + \gamma_{BSL} Z_{j}^L + \tilde{\varepsilon}_{jt}^X
 $$

where $Z^H_j, Z^L_j \perp \tilde{\varepsilon}_{j}^s$
This directly implies our first moment condition: at the true parameter values,  
$$E[\tilde\varepsilon_{jt}^X  Z_{jt}^Y]=0, \quad \forall\: X,Y \in \{H,L\} \times \{H,L\}.$$ 

\end{frame}

\begin{frame}{Housing Supply}

 \begin{itemize}
     \item Recall unobservables for housing supply are construction costs $CC_j$ and interest rate $\iota$  
     \item But we have access to additional observable shifters of housing costs, $x^{reg}$ $x^{geo}$. 
     \item Under the assumption that $x^{reg}$ and $x^{geo}$ and the $Z$s are independent of the unobservables, have moment condition 
$$
E[ \ln (CC_{j}) \psi] = 0 
$$
for all $\psi \in \{ Z^X_{j},  Z_{j}^X x_j^{reg},  Z_{j}^X x_j^{geo}\}$ and for all $X\in\{L,H\}$.

\item Intuition: wage shocks shift labor (population) supply, which affects housing supply via local good consumption
     
 \end{itemize}

 \end{frame}


 \begin{frame}{Labor Supply/Population}

 \begin{itemize} \item Unobservable determinant of labor supply/population decision is exogenous amenity $x_j^A\equiv \xi_j$

 \item Moment condition is 

 $$
E[ \xi_{jt}^z \psi] = 0 
$$
for all $\psi \in \{ Z_{jt}^X,  Z_{jt}^X x_j^{reg},  Z_{jt}^X x_j^{geo}\}$ and for all $X\in\{L,H\}$.

\item Wage shocks influence location choices both directly and indirectly (through prices and amenities) but should be orthogonal to exogenous amenities at fixed parameter values

     
 \end{itemize}
\end{frame}

\begin{frame}{Amenity Supply}

 \begin{itemize}
    \item To identify $\varphi^a$, instrument skill ratio with our instrument  
$$ a_{j} = \varphi^a \log \left(\frac{H_{j}}{L_{j}}\right) + \epsilon^a_j.
$$
\item Under assumption that instruments are uncorrelated with unobserved exogenous changes in city's local amenities, have moment conditions
% Identify supply $\varphi^a$ by instrumenting in the employment ratio with our instrument. That is, we have moment restrictions 
$$
E[\epsilon^a_{j} \psi] = 0$$
where $\psi \in \{ Z^X_{j}, Z_{j}^X x_j^{reg}, Z_{j}^X x_j^{geo}\}$ and for all $X\in\{L,H\}$.

 \end{itemize}

 \end{frame}

 \begin{frame}{Estimation Procedure}
 
\begin{itemize}
\item Linear parameters: 
$$\gamma_{HH}, \gamma_{HL}, \gamma_{LL}, \gamma_{LH}, \alpha_{HH}, \alpha_{HL}, \alpha_{LH}, \alpha_{LL},  \iota, \phi, \phi_{geo}, \phi_{reg}, \sigma_x, \beta^a, \beta^w, \beta^x $$

\item Nonlinear parameters: $\beta^{st}$

\item Estimation procedure a la BLP: 
\begin{itemize}
\item Outer loop: conjecture $\beta^{st}$, and the linear parameters that are not in $\delta$
\item Inner loop: given $\beta^{st}$, first solve for $\delta_j^{z_i}$s using MLE given (3). 
\item Then estimate parameters in $\delta_j^z$ $(\beta^w, \beta^a, \beta^x)$ using  labor supply moment condition $E[\xi_j^z \psi]=0$
\item In outer loop, use all moment conditions and minimize over parameters.
\item Estimate via two-step.

\end{itemize}
\end{itemize}
 \end{frame}
 \end{document}